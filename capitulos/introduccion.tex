\documentclass[../main.tex]{subfiles}
\begin{document}

\section{Introducción}

% Objeto
\subsection{Objeto}
Analizar las posibles alterantivas al propano 
como refrigerante para ciclos de bombas de calor simples de forma teórica y 
experimental que incrementen la eficiencia energética.

% Alcance
\subsection{Alcance}
\#\#\# Fer-ho quan haja acabat el experimental \#\#\#

% Antecedentes y justificación
\subsection{Antecedentes}
% Sistemas convencionales de generación de calor
Actualmente, el 62\% del calor generado para la calefacción de los 
edificios es a base de combustibles fósiles, un 26\% a base de biomasa 
y un 12\% a base de renovables modernas (bombas de calor, termosolar, 
geotermia, etc.), estas proporciones se decantan todavía más 
al uso de combustibles fósiles en la calefacción urbana, con un 
90\% de instalaciones dependiondo de estas (IRENA 2023) \cite{IRENA2023}.

% Bombas de calor frente a calderas
Las bombas de calor podrían reducir el consumo eléctrico y 
el impacto ambiental notablemente (Mayer 2024) \cite{Meyer_2024}.

% Funcionamiento de bomba de calor
Las bombas de calor por compresión se basan en la transferencia de 
calor de un lugar a otro. El ciclo hace circular un refrigerante 
por su interior, y, de forma cíclica y cerrada se evapora, comprime, 
condensa y expande absorbiendo y liberando calor en el acto.

\begin{figure}[h]
    \centering
    \includegraphics[width=200pt]{imagenes/ciclo_basico.png}
    \caption{Ciclo básico de compresión de vapor}
    \label{fig:ciclo_basico}
\end{figure}

Como se puede observar en la Figura \ref{fig:ciclo_basico} este 
ciclo tiene cuatro componentes principales: un evaporador, 
un compresor, un condensador y un dispositivo de expansión. De 
este se puede aprovechar el calor cedido por el condensador o 
el calor absorbido por el evaporador tanto para calentar como 
para enfriar respectivamente, lo que lo hace muy útil tanto a 
nivel doméstico como a nivel industrial.

% Problemas bombas de calor
Actualmente, solamente el sector de la calefacción en los edificios 
mediante bombas de calor presenta un 10\% de las emisiones de 
\begin{math} \left[\text{CO}_{\text{2}}\right]_{\text{eq}} \end{math} 
a la atmósfera (sin considerar otros gases de efecto invernadero). 
De estas emisiones, un tercio son emisiones directas, es decir, 
provienen de la liberación de refrigerantes con alto potencial 
de calentamiento global a la atmósfera, mientras que los otros 
dos tercios están asociados a emisiones indirectas, es decir, 
que están asociadas al consumo eléctrico (IEA 2022) \cite{IEA2022FutureHeatPumps}.

% Marco histórico-regulativo
\subsection{Marco histórico-regulativo}


% Motivación y objetivos
\subsection{Motivación y objetivos}

\newpage
\end{document}