\documentclass[../main.tex]{subfiles}
\begin{document}

\section{Screening termodinámico}
\subsection{Descripción del modelo}
El software usado para el cálculo de las propiedades termodinámicas 
ha sido Refprop v.10.0 \cite{refprop}

El primer paso en este Trabajo de Final de Grado (TFG) involucra 
realizar un \textit{screening} termodinámico para identificar las 
posibles mezclas que superen al propano (R-290) en rendimiento 
energético.

El \textit{screening} consistirá en la combinación de los posibles 
refrigerantes de trabajo en mezclas de tres en tres. Con cada mezcla 
se calcularán las propiedades termodinámicas del ciclo de compresión 
básico con incrementos y decrementos en las proporciones de los 
refrigerantes de un 5\%, resultando en un total de 231 posibles 
combinaciones para cada tres refrigerantes.

El comportamiento de cada combinación se simulará considerando su 
operación en un ciclo básico de vapor. Este involucrará un compresión 
con un rendimento isentrópico fijo, una transformación isobárica en 
cada intercambiador de calor y una expansión isentálpica en la válvula 
de expansión. Cada simulación se lleva a cabo en cuatro escenarios 
diferentes que tipifica la normativa UNE-EN 14511-3 \cite{une14511}
para ensayos de máquinas de calor agua-agua.

% CONDICIONES DE SIMULACIÓN
\subsection{Condiciones de simulación}
Para el cálculo termodinámico del ciclo las principales variables tomarán 
los siguientes valores para cada uno de los escenarios:

Para todos los ensayos: 
\begin{equation*}
SUB = 1\,^{\circ}\mathrm{C},\ SH = 5\,^{\circ}\mathrm{C},\
Ap_{k} = 6.5\,^{\circ}\mathrm{C}, \ Ap_{O} = 3\,^{\circ}\mathrm{C}
\end{equation*}
\begin{equation*}
T_{cw\_in} = 0\,^{\circ}\mathrm{C},\ T_{cw\_out} = -3\,^{\circ}\mathrm{C},\ 
\eta_{s} = 60\%
\end{equation*}

Dependiendo del ensayo:
\begin{itemize}
    \item \textbf{Temperatura baja:} $T_{hw\_in} = 30\,^{\circ}\mathrm{C},\ 
    T_{hw\_out} = 35\,^{\circ}\mathrm{C}$
    \item \textbf{Temperatura intermedia:} $T_{hw\_in} = 40\,^{\circ}\mathrm{C},\ 
    T_{hw\_out} = 45\,^{\circ}\mathrm{C}$
    \item \textbf{Temperatura media:} $T_{hw\_in} = 47\,^{\circ}\mathrm{C},\ 
    T_{hw\_out} = 55\,^{\circ}\mathrm{C}$
    \item \textbf{Temperatura alta:} $T_{hw\_in} = 55\,^{\circ}\mathrm{C},\ 
    T_{hw\_out} = 65\,^{\circ}\mathrm{C}$
\end{itemize}

% ECUACIONES DE CÁLCULO
\subsection{Ecuaciones de cálculo}
La presión de condensación se ha encontrado usando tanto la temperatura de 
entrada del fluido, como el approach de temperatura y el subcooling:
\begin{equation}
p_{k} = f(t = t_{hw\_in} + Ap_{k} + SUB,\ x_{v} = 0)
\end{equation}
Una vez conocido el punto termodinámico de la salida del condensador se puede 
hallar de forma similar el punto posterior a la válvula de expansión:
\begin{equation}
p_{O} = f(t = t_{cw\_out} - Ap_{O},\ h = h_{k,out})
\end{equation}
A partir de estas presiones se pueden hallar un par de magnitudes que definan 
termodinámicamente los puntos de salida del condensador y de entrada del 
evaporador.
\begin{gather}
\mathrm{Punto}_{k,out} = f(p = p_{k},\ t = t_{hw\_in} + Ap_{k})
\label{eq:punto_k,out}
\\
\mathrm{Punto}_{O,in} = f(p = p_{O},\ h = h_{k,out})
\label{eq:punto_O,in}
\end{gather}
Obteniendo la temperatura de vapor saturado en el evaporador se puede hallar 
el punto de salida del evaporador:
\begin{gather}
t_{vap,sat} = f(p = p_{O},\ x_{v} = 1)
\label{eq:t_vap_sat}
\\
\mathrm{Punto}_{O,out} = f(p = p_{O},\ t = t_{vap,sat})
\label{eq:punto_O,out}
\end{gather}
Sabiendo el rendimiento isentrópico del compresor se puede hallar el punto 
a la salida de este, que el mismo que el de la entrada del condensador:
\begin{gather}
h_{c,out,s} = f(p = p_{k},\ s = s_{c,in}),\quad
\label{eq:h_c,out,s}
\\
h_{c,out} = h_{c,in} + \frac{h_{c,out,s} - h_{c,in}}{\eta_{s}}
\label{eq:h_c,out}
\\
\mathrm{Punto}_{k,in} = f(p = p_{k},\ h = h_{c,out})
\label{eq:punto_k,in}
\end{gather}

A continuación se muestran las expresiones que se usan para calcular los 
parámetros del ciclo termodinámico:
\begin{gather}
Glide_{O}=t(p = p_{O},\ x_{v} = 1) - t(p = p_{O},\ x_{v} = x_{v,O,in})
\label{eq:glide_O}
\\
Glide_{k}=t(p = p_{k},\ x_{v} = 1) - t(p = p_{k},\ x_{v} = 0)
\label{eq:glide_k}
\\
VCC = \frac{h_{c,out} - h_{c,in}}{\nu_{suc}}
\label{eq:VCC}
\\
COP = \frac{h_{O,out}-h_{O,in}}{h_{k,in}-h_{k,out}}
\label{eq:COP}
\\
\frac{\dot{m}_{hw}}{\dot{m}_{ref}}=\frac{h_{O,in}-h_{O,out}}
{c_{p}\cdot \left( t_{hw,out}-t_{hw,in} \right)}
\label{eq:caudales}
\\
t_{hw,pinch} = t_{hw,out} - \frac{\dot{m}_{ref}}{\dot{m}_{hw}\cdot c_{p}} \cdot 
\left( h_{k,in} - h(p = p_{k},\ x_{v} = 1)\right)
\label{eq:t_hw_pinch}
\\
t_{pinch} = t(p = p_{k},\ x_{v} = 1) - t_{hw,pinch}
\label{t_pinch}
\end{gather}

% RESTRICCIONES
\subsection{Restricciones}
El proceso incluye filtros que excluyen las mezclas que no cumplen los 
siguientes requisitos:
\begin{itemize}
    \item Glide de condensación ($Glide_{k}$) y glide efectivo de evaporación 
    ($Glide_{O}$) menor a $10\,^{\circ}\mathrm{C}$ ya que la transferencia de 
    calor se reduce al pasar estos umbrales.
    \item Temperatura de descarga del compresor menor a $130\,^{\circ}\mathrm{C}$
    ya que este no está pensado para soportar temperaturas tan elevadas.
    \item Presión de descarga del compresor menor a $25\ \mathrm{bar}$ ya que esta es
    su presión máxima de operación.
    \item Valores de $VCC$ que estén en el intervalo $\pm 30\%\ VCC_{R-290}$  
    para ser compatibles con los compresores de propano.
\end{itemize}
Además, si la diferencia de temperatura entre el fluido y el agua en el 
\textit{pinch point} es menor a $1\,^{\circ}\mathrm{C}$
se volverá a calcular el ciclo pero aumentando en $0.5\,^{\circ}\mathrm{C}$ el 
valor del approach de temperatura del condensador hasta que el ciclo cumpla 
este criterio.

Posteriormente, todas las mezclas que superen los filtros se reevaluarán pero ahora 
con una variación del 1\% en su composición para encontrar la composición que maximiza 
el $COP$

% Resultado bruto
\subsection{Resultado bruto}

% Resultado refinado
\subsection{Resultado refinado}

% Análisis de resultados
\subsection{Análisis de resultados}


\newpage
\end{document}