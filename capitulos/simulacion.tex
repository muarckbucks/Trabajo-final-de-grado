\documentclass[../main.tex]{subfiles}
\begin{document}



\section{Identificación de mezclas}
El primer paso en este Trabajo de Final de Grado (TFG) involucra 
realizar un \textit{screening} termodinámico para identificar las 
posibles mezclas que superen al propano (R-290) en rendimiento 
energético.

El \textit{screening} consistirá en la combinación de los posibles 
refrigerantes de trabajo en mezclas de tres en tres. Con cada mezcla 
se calcularán las propiedades termodinámicas del ciclo de compresión 
básico con incrementos y decrementos en las proporciones de los 
refrigerantes de un 5\%, resultando en un total de 231 posibles 
combinaciones para cada tres refrigerantes.

El comportamiento de cada combinación se simulará considerando su 
operación en un ciclo básico de vapor. Este involucrará un compresión 
con un rendimento isentrópico fijo, una transformación isobárica en 
cada intercambiador de calor y una expansión isentálpica en la válvula 
de expansión. Cada simulación se lleva a cabo en cuatro escenarios 
diferentes que tipifica la normativa \#\#\#\# PONER NORMATIVA \#\#\#\# 
para ensayos de máquinas de calor agua-agua.

Para el cálculo termodinámico del ciclo se asumirán los siguientes valores 
para las siguientes variables: 
\[
SH = 5\,^{\circ}\mathrm{C} \qquad SUB = 1\,^{\circ}\mathrm{C} \qquad
Ap_{k} = 6.5\,^{\circ}\mathrm{C} \qquad Ap_{0} = 3\,^{\circ}\mathrm{C}
\]






\section{Identificación}
Identificación
\end{document}